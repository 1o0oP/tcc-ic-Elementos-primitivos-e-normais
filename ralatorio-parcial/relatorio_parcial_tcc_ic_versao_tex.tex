% Formato do documento
\documentclass[12pt,twoside]{article}

% Pacotes sendo utilizados
\usepackage[utf8]{inputenc}
\usepackage[portuguese]{babel}
\usepackage[T1]{fontenc}
\usepackage{amsmath}
\usepackage{amsfonts}
\usepackage{amsthm}
\usepackage{amssymb}
\usepackage{amscd}
\usepackage{bezier}
\usepackage{latexsym}
\usepackage{mathrsfs}
\usepackage{makeidx}
\usepackage{graphicx}
\usepackage{lmodern}
\usepackage{kpfonts}
\usepackage{cite}
\usepackage{lipsum}
\usepackage{enumerate}
\usepackage{longtable}
\usepackage{hhline}
\usepackage[usenames]{color}
\usepackage{indentfirst}
\usepackage{newlfont}
\usepackage[all]{xy}
\usepackage{setspace}
\usepackage[nottoc, notindex]{tocbibind}
\usepackage{pdfpages}
\usepackage{stmaryrd}
\usepackage{eso-pic}

% Especificações métricas do texto
\usepackage{geometry}
\geometry{
    a4paper,
    left=3cm,
    right=2cm,
    top=3cm,
    bottom=2cm
}
% Indentação do texto
\setlength{\parindent}{1.25cm}

% Faz o "content" ser mencionado como "Sumário"
\renewcommand{\contentsname}{Sumário}

% Criando títulos indexados
\newtheorem{lema}{Lema}[section]
\newtheorem{obs}{Observação}[section]
\newtheorem{teo}{Teorema}[section]

% Início do conteúdo
\begin{document}
  \setstretch{1.5} % Espaçamento entre linhas
  \pagenumbering{gobble} % Cancela a paginação por enquanto

  % Capa do trabalho
  \begin{titlepage}
    \begin{center}
      \parbox{3.5cm}{\includegraphics[scale=0.1]{logo_ufu.png}} \\
      \vspace{1cm}
      {\Large \bf UNIVERSIDADE FEDERAL DE UBERLÂNDIA}
      {\large \bf Bacharelado em Matemática} \\
      \vspace{2cm}
      {\large Luís Henrique da Silva Pinheiro} \\
      \vspace{3cm}
      {\Large \bf Título do trabalho} \\
      {\large \it subtítulo do trabalho} \\
      \vspace{8cm}
      {UBERLÂNDIA \\ Mês e Ano}
    \end{center}
  \end{titlepage}

  \tableofcontents % Cria o sumário
  \thispagestyle{empty} % Retira o estilo padrão do sumário
  \newpage % Quebra a página

  \pagenumbering{arabic} % Começa a enumerar as páginas

  % Primeira seção: Introdução
  \section{Introdução}
    \cite{carothers} \cite{cariello}
    \lipsum[1] \\
    ------------------------------------------------------------------------------------------------------------------
    
    \newpage
  
  % Segunda seção: Desenvolvimento  
  \section{Desenvolvimento}
    \lipsum[1]
        
    \begin{figure}[h!]
      \centering
      \includegraphics[scale=0.5]{funcao_distancia.png}
      \caption{\it Gráfico da função distância ao inteiro mais próximo}
      \label{fig:my_label}
    \end{figure}
    
    \lipsum[2] \\
    ------------------------------------------------------------------------------------------------------------------
    
    \newpage
  
  % Última seção: Conclusão
  \section{Conclusão}
    \lipsum[1] \\
    ------------------------------------------------------------------------------------------------------------------

    \newpage

  % Página de referências
  \pagenumbering{gobble} % Aborta a paginação
  \bibliographystyle{ieeetr} % Configura a forma como vai aparecer as referências
  \bibliography{referencia} % Acessa o arquivo remoto referencia.bib
  \thispagestyle{empty} % Elimina estilo padrão da página de referências

\end{document}