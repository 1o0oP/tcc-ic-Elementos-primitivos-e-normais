\documentclass[12pt,twoside]{article}

\usepackage[utf8]{inputenc}
\usepackage[portuguese]{babel}
\usepackage[T1]{fontenc}
\usepackage{amsmath}
\usepackage{amsfonts}
\usepackage{amsthm}
\usepackage{amssymb}
\usepackage{amscd}
\usepackage{bezier}
\usepackage{latexsym}
\usepackage{mathrsfs}
\usepackage{makeidx}
\usepackage{graphicx}
\usepackage{lmodern}
\usepackage{kpfonts}
\usepackage{cite}
\usepackage{lipsum}
\usepackage{enumerate}
\usepackage{longtable}
\usepackage{hhline}
\usepackage[usenames]{color}
\usepackage{indentfirst}
\usepackage{newlfont}
\usepackage[all]{xy}
\usepackage{setspace}
\usepackage[nottoc, notindex]{tocbibind}
\usepackage{pdfpages}
\usepackage{stmaryrd}
\usepackage{eso-pic}

\usepackage{geometry}
\geometry{
    a4paper,
    left=3cm,
    right=2cm,
    top=3cm,
    bottom=2cm
}

\newtheorem{lema}{Lema}[section]
\newtheorem{obs}{Observação}[section]
\newtheorem{teo}{Teorema}[section]

\setlength{\parindent}{1.25cm} %Indenta o parágrafo

\renewcommand{\contentsname}{Sumário}

\begin{document}

    \setstretch{1.5} % Espaçamento entre linhas
    \pagenumbering{gobble}
    
    \begin{titlepage}
        \begin{center}
            \parbox{3.5cm}{\includegraphics[scale=0.1]{logo_ufu.png}} \\
            \vspace{1cm}
            {\Large \bf UNIVERSIDADE FEDERAL DE UBERLÂNDIA}
            {\large \bf Bacharelado em Matemática} \\
            \vspace{2cm}
            {\large Luís Henrique da Silva Pinheiro} \\
            \vspace{3cm}
            {\Large \bf Título do trabalho} \\
            {\large \it subtítulo do trabalho} \\
            \vspace{8cm}
            {UBERLÂNDIA \\ Mês e Ano}
        \end{center}
    \end{titlepage}
    
    \tableofcontents
    \thispagestyle{empty}
    \newpage
    
    \pagenumbering{arabic}
    
    \section{Introdução}
        \cite{carothers} \cite{cariello}
        \lipsum[1] \\
        ------------------------------------------------------------------------------------------------------------------
        
        \newpage
        
    \section{Desenvolvimento}
        \lipsum[1]
        
        \begin{figure}[h!]
            \centering
            \includegraphics[scale=0.5]{funcao_distancia.png}
            \caption{\it Gráfico da função distância ao inteiro mais próximo}
            \label{fig:my_label}
        \end{figure}
        
        \lipsum[2] \\
        ------------------------------------------------------------------------------------------------------------------
        
        \newpage
        
    \section{Conclusão}
        \lipsum[1] \\
        ------------------------------------------------------------------------------------------------------------------
        
        \newpage

    \pagenumbering{gobble}
    \bibliographystyle{ieeetr}
    \bibliography{referencia}
    \thispagestyle{empty}

\end{document}