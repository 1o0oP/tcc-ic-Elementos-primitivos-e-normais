% Formato do documento
\documentclass[12pt,twoside]{article}

% Pacotes sendo utilizados
\usepackage[utf8]{inputenc}
\usepackage[portuguese]{babel}
\usepackage[T1]{fontenc}
\usepackage{amsmath}
\usepackage{amsfonts}
\usepackage{amsthm}
\usepackage{amssymb}
\usepackage{amscd}
\usepackage{bezier}
\usepackage{latexsym}
\usepackage{mathrsfs}
\usepackage{makeidx}
\usepackage{graphicx}
\usepackage{lmodern}
\usepackage{kpfonts}
\usepackage{cite}
\usepackage{lipsum}
\usepackage{enumerate}
\usepackage{longtable}
\usepackage{hhline}
\usepackage[usenames]{color}
\usepackage{indentfirst}
\usepackage{newlfont}
\usepackage[all]{xy}
\usepackage{setspace}
\usepackage[nottoc, notindex]{tocbibind}
\usepackage{pdfpages}
\usepackage{stmaryrd}
\usepackage{eso-pic}

% Especificações métricas do texto
\usepackage{geometry}
\geometry{
    a4paper,
    left=3cm,
    right=2cm,
    top=3cm,
    bottom=2cm
}
% Indentação do texto
\setlength{\parindent}{1.25cm}

% Faz o "content" ser mencionado como "Sumário"
\renewcommand{\contentsname}{Sumário}

% Criando títulos indexados
\newtheorem{lema}{Lema}[section]
\newtheorem{obs}{Observação}[section]
\newtheorem{teo}{Teorema}[section]

% Início do conteúdo
\begin{document}
  \setstretch{1.5} % Espaçamento entre linhas
  \pagenumbering{gobble} % Cancela a paginação por enquanto

  % Capa do trabalho
  \begin{titlepage}
    \begin{center}
      \parbox{3.5cm}{\includegraphics[scale=0.1]{logo_ufu.png}} \\
      \vspace{1cm}
      {\Large \bf UNIVERSIDADE FEDERAL DE UBERLÂNDIA}
      {\large \bf Bacharelado em Matemática} \\
      \vspace{2cm}
      {\large Luís Henrique da Silva Pinheiro} \\
      \vspace{3cm}
      {\Large \bf Relatório parcial do trabalho:} \\
      {\large \it Elementos primitivos e normais sobre corpos finitos} \\
      \vspace{8cm}
      {UBERLÂNDIA \\ Fevereiro de 2020}
    \end{center}
  \end{titlepage}

  %\tableofcontents % Cria o sumário
  %\thispagestyle{empty} % Retira o estilo padrão do sumário
  %\newpage % Quebra a página

  \pagenumbering{arabic} % Começa a enumerar as páginas

  % Primeira seção: Introdução
  \section*{Introdução}
    % Primeiro parágrafo
    A teoria dos corpos finitos é um ramo da matemática que veio a tona nos
    últimos cinquenta anos por causa de suas diversas aplicações em vários segmentos
    da ciência, entre eles, análise combinatória, teoria dos códigos, criptografia, entre
    outros. Muitas figuras proeminentes na história da matemática contribuíram para o
    desenvolvimento desta teoria, entre eles podemos citar: Pierre de Fermat (1601-
    1665); Leonhard Euler (1707-1783); Joseph-Louis Lagrange (1736-1813); AndrienMarie Legendre (1752-1833); entre outros. Além disso, segundo R. Lidil e H.
    Niederreiter, autores de uma das referências que utilizaremos \cite{finite-fields-1997}, tal teoria começou
    com os trabalhos de Carl Friedrich Gauss (1777-1855) e Evariste Galois (1811-
    1832), contudo, só veio a se tornar interessante para os matemáticos aplicados nas
    últimas décadas.

    % Segundo parágrafo
    Para que se possa entender melhor este tema, explicaremos aqui mesmo, de
    forma breve, o significado destes conceitos. Começamos com a definição de corpo
    finito, este é qualquer coleção finita e não vazia de elementos, munida de duas
    operações binárias entre esses elementos, uma que se comporta como a adição, e
    outra que se comporta como a multiplicação, e quando falamos adição e
    multiplicação estamos nos referindo àquelas definidas entre números reais. Veremos
    no decorrer do trabalho, por exemplo, que se retirarmos de um corpo finito, o
    elemento neutro da adição, os elementos que sobram formam um grupo cíclico com
    a multiplicação, interessante não? Consequentemente, como todo grupo cíclico, ele
    passa a ter um gerador deste grupo. Ora, estes elementos, geradores destes grupos
    assim formados, é exatamente o que chamamos de 'elementos primitivos'. Já a
    definição de 'elemento normal' é um pouquinho mais elaborada. Primeiro,
    começamos com um corpo finito de característica p, com q elementos (q é um
    natural não nulo). Veremos também que p deverá ser um número natural primo, e
    que o fato de p ser a característica deste corpo implica que a cardinalidade q deverá
    ser uma potência de p. Levando isso em conta, escolha um natural não nulo n, e
    considere uma extensão F de grau n do corpo inicial. Da teoria de corpos finitos
    sabemos que esta extensão é um corpo que contém o primeiro, e que pode ser visto
    como um espaço vetorial de dimensão n (finita) sobre ele, é para a base deste
    espaço que olharemos agora. Um elemento x do corpo F é chamado 'elemento normal' quando o conjunto $\{ \ x \ , \ x^{q} \ , \ x^{q^{2}} \ , \ ... \ , \ x^{q^{n-1}} \ \}$ é uma base para este espaço
    vetorial, estas bases assim formadas são chamadas bases normais sobre corpos
    finitos. \\
    %------------------------------------------------------------------------------------------------------------------
    
    %\newpage % Não está sendo preciso quebrar página aqui
  
  % Segunda seção: Objetivos e Relevância  
  \section*{Objetivos e Relevância}
    % Primeiro parágrafo
    Nossos objetivos neste trabalho são, primeiramente, através da referência \cite[Finite fields]{finite-fields-1997}, 
    explorar a teoria de corpos finitos com foco no estudo dos elementos primitivos e normais, passando
    pelos principais conceitos e teoremas que envolvem este tema, como por exemplo, a
    estrutura de corpos finitos, polinômios sobre corpos finitos, somas exponenciais (de
    Gauss, de Jacobi) e então, com suporte, trazer uma seção inteira dedicada aos
    elementos primitivos e normais sobre corpos finitos (Caracterização, Cardinalidades,
    Ações, etc ...). Nesse primeiro ponto as referências \cite[Tópicos de álgebra]{topicos-de-algebra-1970} 
    e \cite[Abstract algebra]{abstract-algebra-2007} serão usadas como um complemento à referência principal citada no início.
    Em seguida, com o intuito de obter maior aprofundamento, voltar esforços para o avanço na direção dos quatro 
    artigos \cite[Primitive normal bases for finite fields]{article-1987}, 
    \cite[Pairs of primitive elements in fields of even order]{article-2014},
    \cite[Existence of some special primitive normal elements over finite fields]{article-2017} e 
    \cite[On primitive normal elements over finite fields]{article-2018} conforme o tempo permitir.
    E por fim, desenvolveremos um texto matemático respeitando os padrões lógico-formais de linguagem que apresente 
    os estudos realizados de forma sucinta e agradável a leitores a nível de graduação.

    % Segundo parágrafo
    O interesse de bases normais sobre corpos finitos decorre tanto da
    curiosidade puramente matemática quanto das aplicações práticas. Com o
    desenvolvimento da teoria de codificação e o surgimento de vários sistemas
    criptográficos utilizando corpos finitos, o trabalho nesta área resultou em vários
    projetos de implementação de hardware's e software's. Estes produtos são
    baseados em esquemas de multiplicação usando bases normais para representar
    corpos finitos, assim é necessário desenvolver uma aritmética de corpos finitos para
    que se possa construir os algorítimos apropriados. É claro que as vantagens de se
    utilizar uma representação de base normal são conhecidas há muitos anos. A
    complexidade do desenho de hardware de tais esquemas de multiplicação é
    fortemente dependente da escolha das bases normais usadas. Por isso, é essencial
    encontrar bases normais de baixa complexidade. \\
    %------------------------------------------------------------------------------------------------------------------
    
    \newpage % Quebra a página

  % Página de referências
  \pagenumbering{gobble} % Aborta a paginação
  \bibliographystyle{ieeetr} % Configura a forma como vai aparecer as referências
  \bibliography{referencia} % Acessa o arquivo remoto referencia.bib
  \thispagestyle{empty} % Elimina estilo padrão da página de referências

\end{document}