% Formato do documento
\documentclass[12pt,twoside]{article}

% Pacotes sendo utilizados
\usepackage[utf8]{inputenc}
\usepackage[portuguese]{babel}
\usepackage[T1]{fontenc}
\usepackage{amsmath}
\usepackage{amsfonts}
\usepackage{amsthm}
\usepackage{amssymb}
\usepackage{amscd}
\usepackage{bezier}
\usepackage{latexsym}
\usepackage{mathrsfs}
\usepackage{makeidx}
\usepackage{graphicx}
\usepackage{lmodern}
\usepackage{kpfonts}
\usepackage{cite}
\usepackage{lipsum}
\usepackage{enumerate}
\usepackage{longtable}
\usepackage{hhline}
\usepackage[usenames]{color}
\usepackage{indentfirst}
\usepackage{newlfont}
\usepackage[all]{xy}
\usepackage{setspace}
\usepackage[nottoc, notindex]{tocbibind}
\usepackage{pdfpages}
\usepackage{stmaryrd}
\usepackage{eso-pic}

% Especificações métricas do texto
\usepackage{geometry}
\geometry{
    a4paper,
    left=3cm,
    right=2cm,
    top=3cm,
    bottom=2cm
}
% Indentação do texto
\setlength{\parindent}{1.25cm}

% Faz o "content" ser mencionado como "Sumário"
\renewcommand{\contentsname}{Sumário}

% Criando títulos indexados
\newtheorem{lema}{Lema}[section]
\newtheorem{obs}{Observação}[section]
\newtheorem{teo}{Teorema}[section]

% Início do conteúdo
\begin{document}
  \setstretch{1.5} % Espaçamento entre linhas
  \pagenumbering{gobble} % Cancela a paginação por enquanto

  % Capa do trabalho
  \begin{titlepage}
    \begin{center}
      \parbox{3.5cm}{\includegraphics[scale=0.1]{logo_ufu.png}} \\
      \vspace{1cm}
      {\Large \bf UNIVERSIDADE FEDERAL DE UBERLÂNDIA}
      {\large \bf Bacharelado em Matemática} \\
      \vspace{2cm}
      {\large Luís Henrique da Silva Pinheiro} \\
      \vspace{3cm}
      {\Large \bf Relatório parcial do trabalho:} \\
      {\large \it Elementos primitivos e normais em corpos finitos} \\
      \vspace{8cm}
      {UBERLÂNDIA \\ Fevereiro de 2020}
    \end{center}
  \end{titlepage}

  %\tableofcontents % Cria o sumário
  %\thispagestyle{empty} % Retira o estilo padrão do sumário
  %\newpage % Quebra a página

  \pagenumbering{arabic} % Começa a enumerar as páginas

  % Seção: Introdução ao tema
  \section*{Introdução}
    % Primeiro parágrafo: Introdução ao assunto
    A teoria dos corpos finitos é um ramo da matemática que veio a tona nos
    últimos cinquenta anos por causa de suas diversas aplicações em vários segmentos
    da ciência, entre eles, análise combinatória, teoria dos códigos, criptografia, entre
    outros. Muitas figuras proeminentes na história da matemática contribuíram para o
    desenvolvimento desta teoria, entre eles podemos citar: Pierre de Fermat (1601-
    1665); Leonhard Euler (1707-1783); Joseph-Louis Lagrange (1736-1813); AndrienMarie Legendre (1752-1833); 
    entre outros. Além disso, segundo R. Lidil e H.
    Niederreiter, autores de uma das referências que utilizaremos \cite{finite-fields-1997}, tal teoria começou
    com os trabalhos de Carl Friedrich Gauss (1777-1855) e Evariste Galois (1811-
    1832), contudo, só veio a se tornar interessante para os matemáticos aplicados nas
    últimas décadas. \\
    %------------------------------------------------------------------------------------------------------------------
    
    %\newpage % Não está sendo preciso quebrar página aqui

  % Seção: Conceitos principais
  \section*{Conceitos principais}
    % Segundo parágrafo: Introdução aos conceitos
    Para que se possa entender melhor este tema, explicaremos aqui mesmo, de
    forma breve, o significado destes conceitos. Começamos com a definição de corpo
    finito, este é qualquer coleção finita e não vazia de elementos, munida de duas
    operações binárias entre esses elementos, uma que se comporta como a adição, e
    outra que se comporta como a multiplicação, e quando falamos adição e
    multiplicação estamos nos referindo àquelas definidas entre números reais. Veremos
    no decorrer do trabalho, por exemplo, que se retirarmos de um corpo finito, o
    elemento neutro da adição, os elementos que sobram formam um grupo cíclico com
    a multiplicação, interessante não? Consequentemente, como todo grupo cíclico, ele
    passa a ter um gerador deste grupo. Ora, estes elementos, geradores destes grupos
    assim formados, é exatamente o que chamamos de 'elementos primitivos'. Já a
    definição de 'elemento normal' é um pouquinho mais elaborada. Primeiro,
    começamos com um corpo finito de característica p, com q elementos (q é um
    natural não nulo). Veremos também que p deverá ser um número natural primo, e
    que o fato de p ser a característica deste corpo implica que a cardinalidade q deverá
    ser uma potência de p. Levando isso em conta, escolha um natural não nulo n, e
    considere uma extensão F de grau n do corpo inicial. Da teoria de corpos finitos
    sabemos que esta extensão é um corpo que contém o primeiro, e que pode ser visto
    como um espaço vetorial de dimensão n (finita) sobre ele, é para a base deste
    espaço que olharemos agora. Um elemento x do corpo F é chamado 'elemento normal' quando o 
    conjunto $\{ \ x \ , \ x^{q} \ , \ x^{q^{2}} \ , \ ... \ , \ x^{q^{n-1}} \ \}$ é uma base 
    para este espaço vetorial, estas bases assim formadas são chamadas bases 
    normais sobre corpos finitos. \\
    %------------------------------------------------------------------------------------------------------------------
    
    %\newpage % Quebra a página

  % Seção: Relevância
  \section*{Relevância}
    % Primeiro parágrafo: Relevância
    O interesse de bases normais sobre corpos finitos decorre tanto da
    curiosidade puramente matemática quanto das aplicações práticas. Com o
    desenvolvimento da teoria de codificação e o surgimento de vários sistemas
    criptográficos utilizando corpos finitos, o trabalho nesta área resultou em vários
    projetos de implementação de hardware's e software's. Estes produtos são
    baseados em esquemas de multiplicação usando bases normais para representar
    corpos finitos, assim é necessário desenvolver uma aritmética de corpos finitos para
    que se possa construir os algorítimos apropriados. É claro que as vantagens de se
    utilizar uma representação de base normal são conhecidas há muitos anos. A
    complexidade do desenho de hardware de tais esquemas de multiplicação é
    fortemente dependente da escolha das bases normais usadas. Por isso, é essencial
    encontrar bases normais de baixa complexidade. \\
    %------------------------------------------------------------------------------------------------------------------
    
    %\newpage % Quebra a página

  % Seção: Objetivos
  \section*{Objetivos}
    % Primeiro parágrafo: Objetivos
    O objetivo por trás deste trabalho é, em primeiro lugar, permitir que o estudante do curso de
    bacharelado em matemática, inscrito para realizar este trabalho, através de leitura, reflexão, resolução
    de exercícios, discussão com orientador e produção de texto lógico formal, se aprofunde no
    desenvolvimento de suas habilidades de pesquisa e autonomia para se tornar mais apto a seguir
    carreira acadêmica, já que este é seu intento. E em segundo lugar, mas não menos importante, criar
    um texto matemático que apresente o tema de forma agradável, apresentando um breve esboço da
    teoria dos corpos finitos, e mostrando como ela pode ser acessível a estudantes a nível de graduação,
    e como ela pode ser aplicada para tratar de elementos primitivos e normais. Em terceiro lugar, uma
    vez que o aluno adquira certa maturidade no tema de estudo, possa colaborar nos projetos nos quais
    o orientador esteja trabalhando no momento.\\
    %------------------------------------------------------------------------------------------------------------------
    
    %\newpage % Quebra a página

  % Seção: Metodologia  
  \section*{Metodologia}
    % Primeiro parágrafo: Metodologia
    O aluno estudará a estrutura de corpos finitos, polinômios sobre corpos finitos e somas
    exponenciais utilizando como referência o texto \cite[Finite fields]{finite-fields-1997} e como livros 
    de apoio os textos \cite[Abstract algebra]{abstract-algebra-2007} e \cite[Tópicos de álgebra]{topicos-de-algebra-1970}. 
    O aluno apresentará semanalmente um seminário de uma hora com o material estudado durante a semana.
    Elementos primitivos e normais serão estudados a partir dos artigos de referências \cite{article-1987}, 
    \cite{article-2014}, \cite{article-2017} e \cite{article-2018}.
    Além disso serão realizadas reuniões semanais de uma hora com o orientador para esclarecer dúvidas
    do aluno. Nos últimos meses da orientação o aluno irá participar dos projetos de pesquisa nos quais
    o orientador estará trabalhando. Serão também realizadas revisões bibliográficas para acrescentar
    conhecimentos atualizados. Mensalmente o aluno irá apresentar os resultados básicos a alunos de
    graduação que estudam temas similares. \\
    %------------------------------------------------------------------------------------------------------------------
    
    %\newpage % Quebra a página

  % Seção: Cronograma  
  \section*{Cronograma}
    % Lista de capítulos e tópicos
    \begin{enumerate}
      \item Estrutura de corpos finitos
      \begin{description}
        \item [(a)] Caracterização dos corpos finitos
        \item [(b)] Raízes de polinômios irredutíveis
        \item [(c)] Traços, normas e bases
        \item [(d)] Raízes da unidade e polinômios ciclotômicos
        \item [(e)] Representação de elementos de corpos finitos
        \item [(f)] Teorema de Wedderburn
      \end{description}
      \item Polinômios sobre corpos finitos
      \begin{description}
        \item [(a)] Ordem de polinômios e polinômios primitivos
        \item [(b)] Polinômios irredutíveis
        \item [(c)] Construção de polinômios irredutíveis
        \item [(d)] Polinômios linearizados
        \item [(e)] Binômios e trinômios
      \end{description}
      \item Somas exponenciais
      \begin{description}
        \item [(a)] Caracteres
        \item [(b)] Somas de Gauss
        \item [(c)] Somas de Jacobi
        \item [(d)] Soma de caracteres com argumentos polinomiais
      \end{description}
      \item Elementos primitivos e normais
      \begin{description}
        \item [(a)] Caracterização de elementos primitivos
        \item [(b)] Número de elementos primitivos em corpos finitos
        \item [(c)] Ação do anel de polinômios sobre um corpo finito
        \item [(d)] Caracterização de elementos normais
        \item [(e)] Número de elementos normais em corpos finitos
        \item [(f)] Conclusão
      \end{description}
    \end{enumerate}  \\
    %------------------------------------------------------------------------------------------------------------------
    
    %\newpage % Quebra a página

  % Seção: Avanço até aqui  
  \section*{Avanço do trabalho até aqui}
    % Primeiro parágrafo: Avanço até aqui
    Até o presente momento o aluno avançou nos estudos seguindo o cronograma proposto, estudou o primeiro capítulo:
    Estrutura de corpos finitos, dentro do qual viu os tópicos: Caracterização dos corpos finitos, Raízes de polinômios
    irredutíveis, Traços, normas e bases, Raízes da unidade e polinômios ciclotômicos, Representação de elementos de
    corpos finitos e Teorema de Wedderburn. E também, o segundo capítulo: Polinômios sobre corpos finitos, onde viu os 
    tópicos: Ordem de polinômios e polinômios primitivos, Polinômios irredutíveis, Construção de polinômios irredutíveis,
    Polinômios linearizados e Binômios e trinômios. Tendo cumprido portanto, aproximadamente $ 50\% $ do cronograma.
    
    % Segundo parágrafo: Próximos passos
    A partir daqui o aluno continuará seguindo com o cronograma, o próximo capítulo a ser estudado será: Somas exponenciais,
    dentro do qual verá os tópicos: Caracteres, Somas de Gauss, Somas de Jacobi e Soma de caracteres com argumentos
    polinomiais. Em seguida, verá o capítulo: Elementos primitivos e normais, no qual passará por: Caracterização de
    elementos primitivos, Número de elementos primitivos em corpos finitos, Ação do anel de polinômios sobre um corpo
    finito, Caracterização de elementos normais e Número de elementos normais em corpos finitos. E por fim passará para
    as conclusões do trabalho, reunindo todos os conhecimentos e iniciando uma breve exploração do artigos citados. \\
    %------------------------------------------------------------------------------------------------------------------
    
    %\newpage % Quebra a página

  % Página de referências
  \pagenumbering{gobble} % Aborta a paginação
  \bibliographystyle{ieeetr} % Configura a forma como vai aparecer as referências
  \bibliography{referencia} % Acessa o arquivo remoto referencia.bib
  \thispagestyle{empty} % Elimina estilo padrão da página de referências

\end{document}